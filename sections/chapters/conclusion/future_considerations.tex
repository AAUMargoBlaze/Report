Due to the structure of how we approached use cases, it can be easy to hand-wave this section in favour of simply
pointing to the next use case.
Instead, we will be looking at complementary features of the language to support the use cases we have laid out.

\subsection{Traversal}\label{subsec:traversal}
A core functionality of graphs is the act of traversing them, without this implemented in the language, it makes little
sense to use the graph data structure over something like an array.
Users of the language should be able to find where nodes lead, and have pathfinding exposed to them either as built in
functionality, or a more advanced function map.

\subsection{Advanced Visualisation}\label{subsec:advanced-visualisation}
The visualisations we have built for the language are made as extension of the graphviz library, however a user of our
language has no mechanism to access anything more advanced than the node names they display.

\subsection{Common Language Features}\label{subsec:common-language-features}
As the use of programming languages has involved, so has expectations for them.
Some expectations we would strive to meet are the following:
\begin{itemize}
    \item \textbf{package management}: Allowing users to share their code and build on top of
    \item \textbf{documentation}: We would thoroughly document syntax and features for people to learn
    \item \textbf{testing}: Most modern software pipelines include testing, our language should support creating them

\end{itemize}