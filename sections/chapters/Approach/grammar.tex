To specify some rules about how a program in Lattice should be composed, we have written the following context-free grammar, using Backus-Naur form.
This grammar can then be used to build a parser and recognize valid Lattice programs and their structure.

\subsubsection*{Global composition of a program}

A program in Lattice would be a sequence of statements and function definitions.

\begin{align*}
    \mathit{Start} \rightarrow \{ \mathit{FunctionDef} \mid \mathit{Statement} \}
\end{align*}

Statements can be variable declarations, variable assignments, graph manipulation (i.e.\ opening a graph context to do operations), print statements, blocks of code in an if-else control structure or a while loop, or a return statement.

\begin{align*}
    \omit $\mathit{Statement}$ \hfil& \rightarrow \mathit{VarDecl} \\
    & \mid \mathit{VarAssignOrGraphManip} \\
    & \mid \mathit{PrintStatement} \\
    & \mid \mathit{IfBlock} \\
    & \mid \mathit{WhileBlock} \\
    & \mid \mathit{ReturnStatement} \\
\end{align*}

\subsubsection*{Variable declaration}

A variable is declared by writing its type followed by its name.
After the variable declarations the statement can end with a semicolon or there can be a value assigned to the variable or graph context can be opened.
This can also be done for a variable that has already been declared.

\begin{align*}
    \omit $\mathit{VarDecl}$ \hfil& \rightarrow \mathit{Type} \: \text{varId} \: \mathit{VarDeclTail} \\
    \omit $\mathit{VarDeclTail}$ \hfil & \rightarrow \mathit{TailVarAssignOrGraphManip} \\
    & \mid \text{semicolon} \\
    \omit $\mathit{VarAssignOrGraphManip}$\hfil & \rightarrow \text{varId} \: \mathit{TailVarAssignOrGraphManip} \\
    \omit $\mathit{TailVarAssignOrGraphManip}$ & \rightarrow \mathit{TailVarAssign} \\
    & \mid  \mathit{TailGraphManip}\ \\
    \omit $\mathit{Type}$ \hfil & \rightarrow \text{typeString} \\
    & \mid \text{typeFloat} \\
    & \mid \text{typeBool} \\
    & \mid \text{typeInt}\\
    & \mid \text{typeGraph} \\
\end{align*} 

\subsubsection*{Variable assignment and arithmetic expressions}
To assign a value to a variable, the variable name needs to be followed by the assignment operator and then the assigned value, which can be a string, a boolean value, an arithmetic expression with variables, numbers and function calls or simply a function call, a number or another variable.

\begin{align*}
    \omit $\mathit{TailVarAssign}$ \hfil & \rightarrow \text{opAssign} \: \mathit{AssignVal} \: \text{semicolon} \\
    \omit $\mathit{AssignVal}$ \hfil & \rightarrow \text{string} \\
    & \mid \mathit{Expr} \\
    & \mid \mathit{BoolVal}\\
    \omit $\mathit{Expr}$ \hfil & \rightarrow \text{opSub} \: \mathit{Expr} \\
    & \mid \mathit{Expr} \: \mathit{MulOp} \: \mathit{Expr} \\
    & \mid \mathit{Expr} \: \mathit{AddOp} \: \mathit{Expr} \\
    & \mid \text{leftParen}\: \mathit{Expr} \: \text{rightParen} \\
    & \mid \mathit{Number} \\
    & \mid \text{varId} \\
    & \mid \mathit{FuncCall} \\
    & \mid \text{KeywordFmap} \: \text{varId} \: \text{varId} \\
    \omit $\mathit{BoolVal}$ \hfil & \rightarrow \text{keywordTrue} \\
    & \mid \text{keywordFalse} \\
    \omit $\mathit{Number}$ \hfil & \rightarrow \text{integer} \\
    & \mid \text{floatLit} \\
    \omit $\mathit{AddOp}$ \hfil & \rightarrow \text{opAdd} \\
    & \mid \text{opSub} \\
    \omit $\mathit{MulOp}$ \hfil & \rightarrow \text{opMult} \\
    & \mid \text{opDiv} \\
    & \mid \text{opRem} \\
\end{align*}

This definition of the grammar doesn't allow the precedence of arithmetic operations to be handled during the building of the parse tree.
However, this is solved with the tool we used to implement our parser.

\subsubsection*{Graph Manipulation}

After having specified the graph context in which we want to work, we can do graph-specific operations as well as the other ones.

\begin{align*}
    \omit $\mathit{TailGraphManip}$ \hfil &\rightarrow \text{leftBrace} \: [\mathit{ListGraphOp}]  \: \text{rightBrace} \\
    \omit $\mathit{ListGraphOp}$ \hfil &\rightarrow \mathit{graphOp} \: \mathit{ListGraphOp} \\
    \omit $\mathit{GraphOp}$ \hfil &\rightarrow \mathit{AddRel}  \\
    & \mid \mathit{AddClone} \\
    & \mid \mathit{AddRef} \\
    & \mid \mathit{Statement} \\
    \omit $\mathit{AddRel}$ \hfil & \rightarrow \text{varId opRelLeft} \: \mathit{Number} \: \text{comma string opRelRight varId} \\
    \omit $\mathit{AddClone}$ \hfil & \rightarrow \text{keywordClone varId semicolon} \\
    \omit $\mathit{AddRef}$ \hfil & \rightarrow \text{keywordRef varId semicolon} \\
\end{align*}

\subsubsection*{Print statement}

The print function is quite minimalist and can only be used for strings and variables, using the keyword $\texttt{print}$.

\begin{align*}
    \mathit{PrintStatement} \rightarrow \text{keywordPrint} \: (\text{varID} \mid \text{string}) \: \text{semicolon}
\end{align*}

\subsubsection*{If-else statement}

The if-else statement structure is quite classic.
After the \texttt{if} keyword, a boolean expression is specified between parenthesis and then the block of code to execute is written between curly brackets.
The following else clause is optional.

\begin{align*}
    \omit $\mathit{IfBlock}$ \hfil &\rightarrow \mathit{IfClause} \: [\mathit{ElseClause}] \\
    \omit $\mathit{IfClause}$ \hfil &\rightarrow \text{keywordIf} \: \text{leftParen} \: \mathit{BoolExpr} \: \text{rightParen} \: \text{leftBrace} \: \{ \mathit{statement}\} \: \text{rightBrace} \\
    \omit $\mathit{ElseClause}$ \hfil &\rightarrow \text{keywordElse} \: \text{leftBrace} \: \{ \mathit{statement} \} \: \text{rightBrace} \\
    \omit $\mathit{BoolExpr}$ \hfil &\rightarrow \text{opBNot} \: \mathit{BoolExpr} \\
    & \mid \mathit{BoolExpr} \: \mathit{BoolOp} \: \mathit{BoolExpr} \\
    & \mid \mathit{AssignVal} \: \mathit{CompOp} \: \mathit{AssignVal} \\
    & \mid \text{leftParen} \: \mathit{BoolExpr} \: \text{RightParen} \\
    & \mid \text{varID} \\
    & \mid \mathit{FuncCall} \\
    & \mid \mathit{BoolVal} \\
    \omit $\mathit{BoolOp}$ \hfil &\rightarrow \text{opBAnd} \\
    & \mid \text{opBOr} \\
    \omit $\mathit{CompOp}$ \hfil & \rightarrow \text{opBEq} \\
    & \mid \text{opBNeq} \\
    & \mid \text{opBGrt} \\
\end{align*}

\subsubsection*{While loop}

The while loop is also very basic.
The statement starts with the keyword \texttt{while} followed by the condition for the loop and then the block of code to repeat.

\begin{align*}
\mathit{WhileBlock} &\rightarrow \text{keywordWhile} \: \text{leftParen} \: \mathit{BoolExpr} \: \text{rightParen} \: \text{leftBrace} \: \{ \mathit{statement}\} \: \text{rightBrace} \\
\end{align*}

\subsubsection*{Function definition and function call}

Function definitions are introduced with the keyword $\texttt{def}$, followed by the name of the function, the types and names of the arguments and then the block of code executed by the function.
A return statement is necessary to write a function, though this rule isn't enforced by the parser.
The function can then be called in the context mentioned above, by writing the function name and the parameters

\begin{align*}
    \omit $\mathit{FuncDef}$ \hfil &\rightarrow \text{keywordDef} \: \mathit{Type} \: \text{varID leftParen} [\mathit{ListArgs}] \text{rightParen} \\
    \hfil \text{leftBrace} \: \{ \mathit{statement} \} \: \text{rightBrace} \\
    \omit $\mathit{ListArgs}$ \hfil &\rightarrow \mathit{Arg} \:  \mathit{TailListArgs} \\
    \omit $\mathit{Arg}$ \hfil &\rightarrow \mathit{Type} \: \text{varId} \\
    \omit $\mathit{TailListArgs}$ \hfil & \rightarrow \{ \text{comma} \: \mathit{Arg} \} \\
    \omit $\mathit{ReturnStatement}$ \hfil &\rightarrow \text{keywordReturn} \: \mathit{AssignVal} \: \text{semicolon} \\
    \omit $\mathit{FuncCall}$ \hfil &\rightarrow \: \text{varID leftParen} [\mathit{ListParam}] \: \text{rightParen} \\
    \omit $\mathit{ListParam}$ \hfil &\rightarrow \mathit{varID} \:  \{ \text{comma} \: \mathit{varID} \} \\
\end{align*}
