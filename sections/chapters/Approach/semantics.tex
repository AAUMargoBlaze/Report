\subsubsection{Variable declaration, assignment, and access}

\begin{itemize}
    \item Each variable must be declared before being assigned a value.
    \item Each variable must have a unique name throughout the program, regardless of the scope.
    \item Variables must be initialized with a value before they are accessed (for example in an expression or function call).
    \item The value assigned to a variable must be compatible with its declared type.
\end{itemize}

\subsubsection{Type rules}

\begin{itemize}
    \item Assignment from float to integer is not allowed, but assignment from integer to float is allowed.
    \item Arithmetic expressions involving integers and floats will result in a float value.
    \item Arithmetic expressions with strings are not allowed.
    \item Boolean expressions can compare floats and integers and the result of \texttt{2.0 == 2} is true.
    \item Comparisons must be made between compatible types.
    \item String comparisons are limited to equality checks; the greater than operator is not applicable.
    \end{itemize}

\subsubsection{Functions}

\begin{itemize}
    \item A function needs to be defined before being called.
    \item The type and number of arguments in a function call must match the declared function signature.
    \item The return statement is only allowed within the scope of a function.
    \item All functions must have unique names.
    \item The value returned by a function must match its declared return type.
    \end{itemize}

\subsubsection{Graph contexts}

\begin{itemize}
    \item A variable needs to be referenced in a context before it can be used in relations or other statements in this context, even if it was first declared in this context.
\item A variable can only be referenced or cloned in a context if it is present (declared or referenced) in the parent context.
    \item Printing a graph will display its nodes and relations as well as the ones in the child graph contexts if there are any.
    \item There can be any number of relation between two nodes, the weight,label and direction of the edge don't have to be unique.
    \item The variable on which the \texttt{fmap} statement is applied must be a graph
    \item The type of all the nodes in a graph must match the one of the parameters of the function called with the \texttt{fmap} keyword.
    \item The function called with the \texttt{fmap} keyword must have only one parameter.
    \end{itemize}


