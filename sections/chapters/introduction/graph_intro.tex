Graphs, in the realm of mathematics and computer science, are powerful structures used to represent and analyze relationships between objects or data points.
At their core, graphs consist of two fundamental components: vertices (also known as nodes) and edges.
Vertices represent individual entities, while edges depict the connections or relationships between them.

Graphs can be further enhanced by incorporating weighted edges, where each edge is assigned a numerical value.
These weights convey additional information about the connections, such as distances, costs, or strengths.
This enrichment enables more nuanced analysis and optimization in a variety of applications.

Adding labels to the edges of a graph can also enhance its interpretation and analysis by providing contextual information, semantic representation, quantitative measures, and facilitating efficient querying and algorithmic applications.
Edge labels contribute to better graph visualization, enable targeted queries, and influence the behavior of algorithms.
They enrich the understanding of relationships between vertices, improve interpretability, and expand the range of applications in domains such as social networks, knowledge graphs, transportation systems, and recommendation engines.

Graphs encompass various substructures that serve specific purposes.
Trees, for instance, are a type of graph with a hierarchical structure, where each vertex (except the root) has a single parent vertex.
Trees are commonly employed in data structures and algorithms, facilitating efficient searching and sorting operations.
Binary trees are a subgroup of trees in which each node have at most two children, resulting in a structure with many interesting properties.

Linked lists, on the other hand, are linear data structures that can be understood as a special type of graph.
Each node in a linked list contains data and a reference (an edge) to the next node, forming a chain-like structure.

With this in mind, we intend to construct a language that is capable of creating and manipulating these graphs, the
language will be called ``Lattice''.