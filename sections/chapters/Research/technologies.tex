In this section we will cover the technologies that will be involved in the final product.

\subsubsection{Language Considerations}
Before creating any code we have two language considerations to make.

The first consideration is the language to use for the compiler;
A good compiler tool is sufficiently portable that it is not difficult to use on any type of machine, historically this
has always been an issue, as Operating Systems and computer architecture have a wide variety - therefore the language
we use must be capable of creating executables for such a wide variety of systems.
In addition, as development time is significant, we should choose a language that provides sufficient tooling to
minimise unnecessary development time.

Our choice of compiler language is C\#, due to its wide system compatibility and support of the~\cite{ANTLR} which we
will explore later.

Next we have to decide what language to compile to, which entails similar considerations as the user will want an
executable that requires minimal to no tweaking.
For this decision we have taken a slightly different approach - in order to fully leverage the language that we compile
to, we plan to support code injection - for this to remain convenient it's better to choose a language that more closely
resembles the one we're translating from.

In this train of thought, we have decided to compile to Python, as it is a similarly high level language to the one
we are designing.

\subsubsection{Compiler Tools}\label{subsubsec:compilertools}
Our choice for compiler tooling follows a similar logic to most development tools, we would like ease of use, while
maintaining support for more complex functionality.

The main decision that had to be made was between ANTLR and leveraging the toolkit that intellij
provides~\cite{IntellijLanguage}.
After choosing ANTLR, we reasoned that we could take advantage of the parse tree that ANTLR creates for both debugging
and traversal properties - as ANTLR creates an easily interpretable image displaying how the grammar operates.
Furthermore, ANTLR introduces with it the listener pattern~\cite[p18]{ANTLRReference}

The document describing the syntax and semantics of a language is referred to as a ``grammar''\cite{Grammars}, both
ANTLR and the intellij plugin are based on Extended Backus-Naur Form (EBNF), they also both include a feature for
regex injection - allowing more well-formed rules to be written using fewer lines of code.

\subsubsection{Language Tools}
For the purpose of having usable output, we have decided to visualise the graphs we build.
The tool we'll be using for this purpose is GraphViz\cite{GraphViz} - which takes advantage of standardised grammar
called the DOT language, that can be used to consistently generate the same image.
Graphviz has a python library, so we can easily incorporate it into the print statements we define.