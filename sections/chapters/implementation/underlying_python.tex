Our goal for the python implementation is to find a way that is both convenient to write and efficient to execute.
To achieve this we will build a python library that can abstract graph building concepts to simplify the process for us.

Some early considerations to make are as follows:
\begin{itemize}
    \item {Data should be loaded into memory sparingly, allowing for more ambitious implementations}
    \item {Graphs must be viewable, this allows for easy debugging}
    \item {Building graphs should be a self-documenting process, as convenient and easy to read as possible
        (this will allow easier code generation later)}
\end{itemize}


Our first attempt at improving efficiency is to avoid forcing graphs to maintain the location of each node in memory.
The way we approach this is by having `exposed' and `hidden' nodes, in essence, a graph does not need to be aware
of the nodes it has, as we can use pathfinding to expose these nodes.
For larger graphs this allows us to define several key points that pathfinding should originate from (for example,
a pathfinding algorithm could expose nodes in city centres, then begin traversal from there).
Hidden nodes refer to nodes that have edges with nodes that are exposed within the graph, and are only reachable by
pathfinding algorithms, or other graphs that expose those specific nodes.

\begin{lstlisting}[caption={Adding nodes to a graph},captionpos=b,label={lst:hidden-visible-nodes}]
    def add_nodes(self, **nodes: Node):
        """Exposed nodes are defined in key value pairs"""
        # add regular nodes
        for key, value in nodes.items():
            self._nodes[key] = value

    def add_node(self, node: Node, label):
        """Create a node within the graph"""
        self._nodes[label] = node
\end{lstlisting}

Relationships between nodes are an equally important consideration in the construction of graphs.
Earlier in the use cases section we touched on the considerations when creating edges that can account for any type of
use case.
Our eventual approach is to create a basic Edge that at most defines the relationship, and use inheritance to attach
functionality to such edges.
Additionally, this will allow us to traverse edges depending on their ``type'' - for example we can build a traversal
edge that inherits regular edge and adds the concept of a cost; we can then build algorithms that explore the graph
only on edges that are a child of a traversal edge.
Our approach allows us to traverse graphs in different ways for different situations, for example we may want to
construct a graph that has hidden computational edges, then for debugging purposes we can connect nodes using
visual edges.
As edges are a facet of the nodes themselves, creating edges is a function of the predecessor node - connecting it to a
successor node.

With a basic approach defined we can start working on the visualisation.
As an attempt to maintain the convenience we've been focusing on, we want a visualisation to be as simple as using a
python print statement - which will also simplify code generation later.
Python allows us to override the way that classes are interacted with by exposing the base functionality as an
inherited trait~\cite{PythonDunder} -this is significant as invoking a print statement on an object invokes the special
\textbf{\_\_str\_\_} method and returns the output to the console.
For our intended functionality, we override this method to simply return the unicode zero width space (returning no
output to the standard out).
Rather than printing to standard out, we have decided that the graph structure should be shown as a visual
representation, for this we will output an image once the print statement is called.


\begin{lstlisting}[caption={Visualising graphs},captionpos=b,label={lst:python-visualise-graph}]
    def visualise(self, fingerprint: str = None):
        """
        Display visible nodes in the graph
        Node names are set by the label of the node, rather than the key set by the graph
        """
        # Create a new fingerprint for each visualisation to generate a single file per print
        runtime_fingerprint = fingerprint or uuid.uuid4()
        dot = graphviz.Digraph(str(runtime_fingerprint))
        explored = []
        for name, node in self.nodes.items():
            if node not in explored:
                explored = self._recursive_visualiser(node, explored, dot)

        dot.render(directory='dot_output', view=True)

    def _recursive_visualiser(self, node, explored, dot):
        """Visualise every node by exploring node connections"""
        explored.append(node)
        shape = 'circle'
        if node in self._nodes.values():
            shape = 'doublecircle'
        dot.node(str(node), node.get_label(), shape=shape)
        for edge, next_node in node.successors:
            if isinstance(edge, VisualEdge):
                dot.edge(str(node), str(next_node), label=str(edge.get_label()))
                if next_node not in explored:
                    explored = self._recursive_visualiser(next_node, explored, dot)

        return explored

    def __repr__(self):
        self.visualise()
        return "​"
\end{lstlisting}

To avoid visualising too many edges in larger graphs, we create a ``Visual Edge'' class, any edge that inherits this
will be an edge that can be followed during visual exploration.

Regarding visualisations, we have already mentioned that the technology we will be using is GraphViz, which will store
the graphs that users visualise in `dot' notation (allowing graph visuals to remain consistent).
Graph building in graphviz consists of defining individual nodes and connecting them via edges - similar to the
construction of regular graphs.
Naturally GraphViz exposes many visual components, to avoid complicating the visualisation process we will focus on
only exposing the labels for both nodes and edges.

Consequently, we have nodes and edges that abuse inheritance to abstract the features we have laid out in the use cases
section.

An additional feature that we would like to implement in the Lattice language is the ability to map a function (FMAP) on
to  every node in graph, we have decided that the most convenient place to do this is within the python library itself.
Our first FMAP approach will be a simple ``flat'' application of the function on to each node - the given function will
perform a transformation of that node's data.
More complex FMAP's would involve contextual knowledge of traversal path and edges, however we have considered that out
of scope for the basic implementation.

\begin{lstlisting}[caption={FMAP on a graph including examples},captionpos=b,label={lst:python-fmap-graph}]
    def apply(self, func):
        """Apply a function flatly to every node in the network"""
        explored = []
        for node in self.nodes.values():
            if node not in explored:
                explored = self._recursive_apply(node, func, explored)

    def _recursive_apply(self, node: Node, func: Callable, explored: List[Node]):
        """Apply the function to this node then call this function on every connected node"""
        explored.append(node)
        node.data = func(node.data)
        for edge, next_node in node.successors:
            if next_node not in explored:
                self._recursive_apply(next_node, func, explored)

        return explored
    ##################### APPLICATION #####################
    simple_graph = Graph()
    simple_graph.add_nodes(node1=Node(1))
    node2 = Node(2)
    node3 = Node(3)
    node4 = Node(4)
    simple_graph.get_node('node1').add_edge(Edge(), node2)
    node2.add_edge(Edge(), node3)
    node3.add_edge(Edge(), node4)

    simple_graph.apply(lambda x: x + 1)
    simple_graph.apply(print)
    ####################### OUTPUT ########################
    2
    3
    4
    5
\end{lstlisting}

Lastly, for some operations we would like to perform a full copy of the graph and its nodes.
This should be an absolutely last resort action, as any implementation requires an expensive full exploration of the
graph (including hidden nodes).
A deepcopy in our implementation will go over every single node that the graph contains and is connected to, and
constructs a copy of it.

\begin{lstlisting}[caption={Deep copying a graph},captionpos=b,label={lst:python-deepcopy-graph}]
    def deepcopy(self):
        """Return a network of nodes encased by this graph, but copied"""
        new_graph = Graph()
        explored = {}
        for name, node in self.nodes.items():
            if node not in explored:
                new_node = node.copy(label=node.get_label())
                explored[node] = new_node
                new_graph.add_nodes(**{name: new_node})
                explored = self._recursive_copier(new_node, node, explored, new_graph)

        return new_graph

    def _recursive_copier(self, new_node, frontier_node, explored, graph):
        """Explore each new node and copy it into a given graph"""
        for edge, next_node in frontier_node.successors:
            if next_node not in explored:
                next_new_node = next_node.copy(label=next_node.get_label())
                explored[next_node] = next_new_node
                if next_node in self.nodes.values():
                    node_name = list(self._nodes.keys())[list(self._nodes.values()).index(next_node)]
                    graph.add_nodes(**{node_name: next_new_node})

                explored = self._recursive_copier(next_new_node, next_node, explored, graph)

            new_node.add_edge(edge, explored[next_node])
        return explored
\end{lstlisting}

We can now create such a library and upload it to the public PyPi repository to allow for easier execution of compiled
code~\cite{LatticePyPi}.