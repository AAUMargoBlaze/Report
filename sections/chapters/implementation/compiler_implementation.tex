This section will cover the internal logic of the compiler.

\subsection{Listeners}

\subsection{Exceptions}

\subsection{Testing}

\subsection{Distribution}
Our last task is to ensure that end users are capable of conveniently executing the code we have provided.
Leveraging the use of libraries helps us streamline the process, but also introduces steps that an end user would have
to follow.
Ideally our implementation would execute directly without many requirements, however, without having a clear way to do
that we have decided to settle on using a Docker image as an executable~\cite{DockerExec}, which will limit the usage
requirements to just running docker.

Unfortunately the nature of graphviz visualisations rely on specific subsystems to provide an image on screen, naturally
these subsystems do not communicate with those of the host machine without tweaking, which would defeat the purpose of
using docker for convenience.
Rather than attempt a complex solution, we have decided to rely on the ``libgraph-easy-perl'' package, which can
translate dot notation to ascii form - this can then be printed to standard out.

\begin{lstlisting}[caption={The DockerFile used for distribution},captionpos=b,label={lst:dockerfile}]
    FROM python:3.9

    RUN pip install graphviz
    RUN pip install lattice-graph-manipulation
    RUN apt update
    RUN apt install -y graphviz
    RUN apt install -y xdg-utils
    RUN apt install -y libgraph-easy-perl

    ENTRYPOINT ["python"]
\end{lstlisting}

We can now simply use environment variables to ensure that the text version is printed for distributed versions of
the application.
